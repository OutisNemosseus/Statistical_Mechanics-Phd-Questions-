% Problems 1086-1100: Thermodynamic Functions (continued)

\section{Problems 1086-1100}

These problems cover advanced applications of thermodynamic functions including:

\begin{itemize}
\item Adiabatic demagnetization
\item Joule-Thomson effect
\item Throttling processes
\item Thermodynamic stability conditions
\end{itemize}

%% PROBLEM 1095 %%
\section{Problem 1095: Adiabatic Demagnetization}

A paramagnetic salt is used for cooling by adiabatic demagnetization.

\subsection*{Solution}

The entropy of a paramagnetic substance can be written as:
\begin{equation}
S = S(T, H)
\end{equation}

For an adiabatic process ($dS = 0$):
\begin{equation}
\left(\frac{\partial T}{\partial H}\right)_S = -\frac{(\partial S/\partial H)_T}{(\partial S/\partial T)_H}
\end{equation}

Using Curie's law $M = CH/T$ and thermodynamic relations:
\begin{equation}
T_f = T_i \frac{H_f}{H_i}
\end{equation}

This shows that reducing the magnetic field adiabatically reduces the temperature.

%% PROBLEM 1097-1100 %%
\section{Problems 1097-1100: Atmospheric Thermodynamics}

These problems deal with:
\begin{itemize}
\item Pressure distribution in atmosphere
\item Adiabatic lapse rate
\item Convective equilibrium
\end{itemize}

\subsection*{Adiabatic Lapse Rate}

For an adiabatic atmosphere:
\begin{equation}
\frac{dT}{dz} = -\frac{(\gamma-1)}{\gamma}\frac{mg}{k}
\end{equation}

For air with $\gamma = 7/5$:
\begin{equation}
\frac{dT}{dz} \approx -10 \text{ K/km}
\end{equation}
