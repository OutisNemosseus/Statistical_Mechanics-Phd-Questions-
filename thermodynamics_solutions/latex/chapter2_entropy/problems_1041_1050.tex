% Problems 1041-1050: The Second Law and Entropy

%% PROBLEM 1041 %%
\section{Problem 1041}
\textbf{(MIT)}

A building at a temperature $T$ (in K) is heated by an ideal heat pump which uses the atmosphere at $T_0$ (K) as heat source. The pump consumes power $W$ and the building loses heat at a rate $\alpha(T - T_0)$. What is the equilibrium temperature of the building?

\subsection*{Solution}

Let $T_e$ be the equilibrium temperature. Heat is given out by the pump at the rate $Q_1 = W/\eta$, where $\eta = 1 - T_0/T_e$. At equilibrium $Q_1 = \alpha(T_e - T_0)$, so that:
\begin{equation}
W = \frac{\alpha}{T_e}(T_e - T_0)^2
\end{equation}

from which we get:
\begin{equation}
\boxed{T_e = T_0 + \frac{W}{2\alpha} + \sqrt{T_0\frac{W}{\alpha} + \left(\frac{W}{2\alpha}\right)^2}}
\end{equation}

%% PROBLEM 1042 %%
\section{Problem 1042}
\textbf{(Wisconsin)}

Let $M$ represent a certain mass of coal which we assume will deliver 100 joules of heat when burned. Assume the plant is ideal (no waste in turbines or generators) discharging its heat at $30^\circ$C to a river. How much heat will $M$, burned at the plant to generate electricity, provide for the house when the electricity is:

(a) delivered to residential resistance-heating radiators?

(b) delivered to a residential heat pump (again assumed ideal) boosting heat from a reservoir at $0^\circ$C into a hot-air system at $30^\circ$C?

\subsection*{Solution}

When $M$ is burned in the power plant (at $1000^\circ$C = 1273 K), the work it provides is:
\begin{equation}
W = Q\left(1 - \frac{T_{\text{cold}}}{T_{\text{hot}}}\right) = 100 \times \left(1 - \frac{303}{1273}\right) = 76.2 \text{ J}
\end{equation}

This is delivered in the form of electric energy.

\textbf{(a)} When it is delivered to residential resistance-heating radiators, it will transform completely into heat:
\begin{equation}
\boxed{Q' = W = 76.2 \text{ J}}
\end{equation}

\textbf{(b)} When the electricity is delivered to a residential heat pump, heat flows from a source of lower temperature to a system at higher temperature, the coefficient of performance being:
\begin{equation}
\text{COP} = \frac{T_{\text{hot}}}{T_{\text{hot}} - T_{\text{cold}}} = \frac{303}{30} = 10.1
\end{equation}

Hence the heat provided for the house is:
\begin{equation}
\boxed{Q' = \text{COP} \times W = 10.1 \times 76.2 = 770 \text{ J}}
\end{equation}

%% PROBLEM 1043 %%
\section{Problem 1043}
\textbf{(CUSPEA)}

An air conditioner is a device used to cool the inside of a home. A home air conditioner operating on a reversible Carnot cycle between the inside, absolute temperature $T_2$, and the outside, absolute temperature $T_1 > T_2$, consumes $P$ joules/sec from the power lines when operating continuously.

(a) Develop a formula for the efficiency ratio $Q_2/P$ in terms of $T_1$ and $T_2$.

(b) Heat leakage into the house follows Newton's law $Q = A(T_1 - T_2)$. Develop a formula for $T_2$ in terms of $T_1$, $P$, and $A$.

(c) Find the highest outside temperature for which it can maintain $20^\circ$C inside.

(d) In the winter, find the lowest outside temperature for which it can maintain $20^\circ$C inside.

\subsection*{Solution}

\textbf{(a)} From the first and second thermodynamic laws, we have:
\begin{equation}
Q_1 = P + Q_2, \quad \frac{Q_2}{T_2} = \frac{Q_1}{T_1}
\end{equation}

Hence:
\begin{equation}
\boxed{\frac{Q_2}{P} = \frac{T_2}{T_1 - T_2}}
\end{equation}

\textbf{(b)} At equilibrium, heat leakage into the house equals the heat transferred out:
\begin{equation}
Q_2 = A(T_1 - T_2)
\end{equation}

Using the result from (a):
\begin{equation}
P \cdot \frac{T_2}{T_1 - T_2} = A(T_1 - T_2)
\end{equation}

Solving:
\begin{equation}
\boxed{T_2 = \frac{T_1}{2} + \frac{1}{2}\left(\frac{P}{A} - \sqrt{\frac{P^2}{A^2} + \frac{4PT_1}{A}}\right)}
\end{equation}

\textbf{(c)} When the air conditioner works 30\% of the time at $T_2 = 293$ K, $T_1 = 303$ K:
\begin{equation}
\frac{P}{A} = \frac{100}{30} \cdot \frac{100}{293} \approx 1.14
\end{equation}

When it operates continuously:
\begin{equation}
T_1 = 293 + 18.26 \text{ K} = \boxed{38.3^\circ\text{C}}
\end{equation}

\textbf{(d)} When the cycle is reversed in winter:
\begin{equation}
T_1 = T_2 - \sqrt{\frac{P}{A} \cdot T_2} = 293 - (1.14 \times 293)^{1/2} = 275 \text{ K} = \boxed{2^\circ\text{C}}
\end{equation}

%% PROBLEM 1044 %%
\section{Problem 1044}
\textbf{(Wisconsin)}

Calculate the change of entropy involved in heating a gram-atomic weight of silver at constant volume from $0^\circ$ to $30^\circ$C. The value of $C_v$ over this temperature may be taken as a constant equal to 5.85 cal/deg$\cdot$mole.

\subsection*{Solution}

The change of entropy is:
\begin{equation}
\Delta S = \int \frac{nC_v dT}{T} = nC_v \ln\frac{T_2}{T_1} = 5.85 \ln\frac{303}{273} = \boxed{0.61 \text{ cal/K}}
\end{equation}

%% PROBLEM 1045 %%
\section{Problem 1045}
\textbf{(Wisconsin)}

A body of constant heat capacity $C_p$ and a temperature $T_i$ is put into contact with a reservoir at temperature $T_f$. Equilibrium between the body and the reservoir is established at constant pressure. Determine the total entropy change and prove that it is positive for either sign of $(T_f - T_i)/T_f$. You may regard $|T_f - T_i|/T_f < 1$.

\subsection*{Solution}

We assume $T_i \neq T_f$. The change of entropy of the body is:
\begin{equation}
\Delta S_1 = C_p \ln\frac{T_f}{T_i}
\end{equation}

The change of entropy of the heat source is:
\begin{equation}
\Delta S_2 = -\frac{C_p(T_f - T_i)}{T_f}
\end{equation}

Therefore the total entropy change is:
\begin{equation}
\Delta S = \Delta S_1 + \Delta S_2 = C_p\left[\ln\frac{T_f}{T_i} - \frac{T_f - T_i}{T_f}\right] = C_p\left[\frac{T_i}{T_f} - 1 - \ln\frac{T_i}{T_f}\right]
\end{equation}

When $x > 0$ and $x \neq 1$, the function $f(x) = x - 1 - \ln x > 0$. Therefore:
\begin{equation}
\boxed{\Delta S > 0}
\end{equation}

%% PROBLEM 1046 %%
\section{Problem 1046}
\textbf{(Wisconsin)}

One kg of H$_2$O at $0^\circ$C is brought in contact with a heat reservoir at $100^\circ$C. When the water has reached $100^\circ$C:

(a) what is the change in entropy of the water?

(b) what is the change in entropy of the universe?

(c) how could you heat the water to $100^\circ$C so the change in entropy of the universe is zero?

\subsection*{Solution}

The process is irreversible. In order to calculate the change of entropy of the water and of the whole system, we must construct a reversible process which has the same initial and final states.

\textbf{(a)} We assume the process is a reversible process of constant pressure. The change in entropy of the water is:
\begin{equation}
\Delta S_{\text{H}_2\text{O}} = \int_{273}^{373} \frac{mC_{\text{H}_2\text{O}} dT}{T} = mC_{\text{H}_2\text{O}} \ln\frac{373}{273}
\end{equation}

Substituting $m = 1$ kg and $C_{\text{H}_2\text{O}} = 4.18$ J/g$\cdot$K:
\begin{equation}
\boxed{\Delta S_{\text{H}_2\text{O}} = 1305 \text{ J/K}}
\end{equation}

\textbf{(b)} The change in entropy of the heat source is:
\begin{equation}
\Delta S_{\text{hs}} = -\frac{|Q|}{T} = -\frac{1000 \times 4.18 \times 100}{373} = -1121 \text{ J/K}
\end{equation}

Therefore the change of entropy of the whole system is:
\begin{equation}
\boxed{\Delta S = 1305 - 1121 = 184 \text{ J/K}}
\end{equation}

\textbf{(c)} We can imagine infinitely many heat sources which have infinitesimal temperature difference between two adjacent sources from $0^\circ$C to $100^\circ$C. The water comes in contact with the infinitely many heat sources in turn in the order of increasing temperature. This process which allows the temperature of the water to increase from $0^\circ$C to $100^\circ$C is reversible; therefore $\boxed{\Delta S = 0}$.

%% PROBLEM 1047 %%
\section{Problem 1047}
\textbf{(UC Berkeley)}

Compute the difference in entropy between 1 gram of nitrogen gas at a temperature of $20^\circ$C and under a pressure of 1 atm, and 1 gram of liquid nitrogen at a temperature $-196^\circ$C, which is the boiling point of nitrogen, under the same pressure of 1 atm. The latent heat of vaporization of nitrogen is 47.6 cal/gm. Regard nitrogen as an ideal gas with molecular weight 28, and with a temperature-independent molar specific heat at constant pressure equal to 7.0 cal/mol$\cdot$K.

\subsection*{Solution}

The number of moles of 1 g nitrogen is:
\begin{equation}
n = 1/28 = 3.57 \times 10^{-2} \text{ mol}
\end{equation}

The entropy difference of an ideal gas at $20^\circ$C and at $-196^\circ$C is:
\begin{equation}
\Delta S' = nC_p \ln(T_1/T_2) = 0.0357 \times 7.0 \times \ln\frac{293}{77} = 0.33 \text{ cal/K}
\end{equation}

The entropy change at phase transition is:
\begin{equation}
\Delta S'' = \frac{nL}{T_2} = \frac{1 \times 47.6}{77} = 0.62 \text{ cal/K}
\end{equation}

Therefore:
\begin{equation}
\boxed{\Delta S = \Delta S' + \Delta S'' = 0.95 \text{ cal/K}}
\end{equation}

%% PROBLEM 1048 %%
\section{Problem 1048}
\textbf{(SUNY Buffalo)}

A Carnot engine is made to operate as a refrigerator. This refrigerator freezes water at $0^\circ$C and heat from the working substance is discharged into a tank containing water maintained at $20^\circ$C. Determine the minimum amount of work required to freeze 3 kg of water.

\subsection*{Solution}

The Carnot cycle as a refrigerator consists of:
\begin{itemize}
\item 1-2: adiabatic compression
\item 2-3: isothermal compression
\item 3-4: adiabatic expansion
\item 4-1: isothermal expansion
\end{itemize}

The refrigeration coefficient of performance is:
\begin{equation}
\text{COP} = \frac{Q_2}{W} = \frac{T_2}{T_1 - T_2}
\end{equation}

Hence:
\begin{equation}
W = Q_2 \cdot \frac{T_1 - T_2}{T_2}
\end{equation}

$Q_2 = ML$ is the latent heat for $M = 3$ kg of water at $T = 0^\circ$C to become ice. As:
\begin{equation}
L = 3.35 \times 10^5 \text{ J/kg}
\end{equation}

we find:
\begin{equation}
W = 3 \times 3.35 \times 10^5 \times \frac{20}{273} = \boxed{73.4 \text{ kJ}}
\end{equation}

%% PROBLEM 1049 %%
\section{Problem 1049}
\textbf{(Wisconsin)}

$n = 0.081$ kmol of He gas initially at $27^\circ$C and pressure $= 2 \times 10^5$ N/m$^2$ is taken over the path $A \to B \to C$. For He: $C_v = 3R/2$, $C_p = 5R/2$. Assume the ideal gas law.

(a) How much work does the gas do in expanding at constant pressure from $A \to B$?

(b) What is the change in internal energy from $A \to B$?

(c) How much heat is absorbed in going from $A \to B$?

(d) If $B \to C$ is adiabatic, what is the entropy change and what is the final pressure?

\subsection*{Solution}

\textbf{(a)} For $A \to B$, the external work is:
\begin{equation}
W = p_A(V_B - V_A) = \boxed{1.0 \times 10^5 \text{ J}}
\end{equation}

\textbf{(b)} For $A \to B$, the increase of the internal energy is:
\begin{equation}
\Delta U = C_v \Delta T = \frac{C_v p_A(V_B - V_A)}{R} = \frac{3W}{2} = \boxed{1.5 \times 10^5 \text{ J}}
\end{equation}

\textbf{(c)} By the first law of thermodynamics, the heat absorbed during $A \to B$ is:
\begin{equation}
Q = W + \Delta U = \boxed{2.5 \times 10^5 \text{ J}}
\end{equation}

\textbf{(d)} For $B \to C$, the adiabatic process of a monatomic ideal gas satisfies:
\begin{equation}
pV^\gamma = \text{const}, \quad \gamma = C_p/C_v = 5/3
\end{equation}

Thus $p_B V_B^\gamma = p_C V_C^\gamma$ and:
\begin{equation}
p_C = (V_B/V_C)^\gamma p_B = \boxed{1.24 \times 10^5 \text{ N/m}^2}
\end{equation}

In the process of reversible adiabatic expansion, the change in entropy is:
\begin{equation}
\Delta S = nC_v \ln\frac{T_C}{T_B} + nR\ln\frac{V_C}{V_B} = nC_v \ln\frac{T_C V_C^{\gamma-1}}{T_B V_B^{\gamma-1}} = \boxed{0}
\end{equation}

%% PROBLEM 1050 %%
\section{Problem 1050}
\textbf{(Wisconsin)}

A mole of an ideal gas undergoes a reversible isothermal expansion from volume $V_1$ to $2V_1$.

(a) What is the change in entropy of the gas?

(b) What is the change in entropy of the universe?

Suppose the same expansion takes place as a free expansion:

(a) What is the change in entropy of the gas?

(b) What is the change in the entropy of the universe?

\subsection*{Solution}

\textbf{Reversible isothermal expansion:}

(a) In the process of isothermal expansion, the external work done by the system is:
\begin{equation}
W = \int p\,dV = RT\ln 2
\end{equation}

Because the internal energy does not change in this process, the work is supplied by the heat absorbed from the external world. Thus the increase of entropy of the gas is:
\begin{equation}
\boxed{\Delta S_1 = \frac{Q}{T} = R\ln 2}
\end{equation}

(b) The change in entropy of the heat source $\Delta S_2 = -\Delta S_1$, thus the total change in entropy of the universe is:
\begin{equation}
\boxed{\Delta S = 0}
\end{equation}

\textbf{Free expansion:}

If it is a free expansion, the internal energy of the system is constant. As its final state is the same as for the isothermal process, the change in entropy of the system is also the same:
\begin{equation}
\boxed{\Delta S_{\text{gas}} = R\ln 2}
\end{equation}

In this case, the state of the heat source does not change, neither does its entropy. Therefore the change in entropy of the universe is:
\begin{equation}
\boxed{\Delta S = R\ln 2}
\end{equation}
