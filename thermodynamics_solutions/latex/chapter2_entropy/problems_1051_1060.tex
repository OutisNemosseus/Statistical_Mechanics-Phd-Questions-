% Problems 1051-1060: The Second Law and Entropy

%% PROBLEM 1051 %%
\section{Problem 1051}
\textbf{(MIT)}

$N$ atoms of a perfect gas are contained in a cylinder with insulating walls, closed at one end by a piston. The initial volume is $V_1$ and the initial temperature $T_1$.

(a) Find the change in temperature, pressure and entropy that would occur if the volume were suddenly increased to $V_2$ by withdrawing the piston.

(b) How rapidly must the piston be withdrawn for the above expressions to be valid?

\subsection*{Solution}

\textbf{(a)} The gas does no work when the piston is withdrawn rapidly. Also, the walls are thermally insulating, so that the internal energy of the gas does not change, i.e., $dU = 0$. Since the internal energy of an ideal gas is only dependent upon temperature $T$, the change in temperature is 0, i.e., $T_2 = T_1$.

As for the pressure: $p_2/p_1 = V_1/V_2$.

The increase in entropy is:
\begin{equation}
\boxed{\Delta S = Nk\ln\frac{V_2}{V_1}}
\end{equation}

\textbf{(b)} The speed at which the piston is withdrawn must be far greater than the mean speed of the gas molecules:
\begin{equation}
\boxed{u \gg \bar{v} = \sqrt{\frac{8kT_1}{\pi m}}}
\end{equation}

%% PROBLEM 1052 %%
\section{Problem 1052}
\textbf{(Wisconsin)}

A cylinder contains a perfect gas in thermodynamic equilibrium at $p$, $V$, $T$, $U$ (internal energy) and $S$ (entropy). The cylinder is surrounded by a very large heat reservoir at the same temperature $T$. The piston is moved to produce a small volume change $\pm\Delta V$. For each of the five processes below show whether the changes in the other quantities have been positive, negative, or zero.

\begin{center}
\begin{tabular}{|l|c|c|c|c|}
\hline
Process & $\Delta T$ & $\Delta U$ & $\Delta S$ & $\Delta p$ \\
\hline
1. $(+\Delta V)$ (slow) (conduct) & 0 & 0 & $+$ & $-$ \\
2. $(+\Delta V)$ (slow) (insulate) & $-$ & $-$ & 0 & $-$ \\
3. $(+\Delta V)$ (fast) (insulate) & 0 & 0 & $+$ & $-$ \\
4. $(+\Delta V)$ (fast) (conduct) & 0 & 0 & $+$ & $-$ \\
5. $(-\Delta V)$ (fast) (conduct) & 0 & 0 & $-$ & $+$ \\
\hline
\end{tabular}
\end{center}

\subsection*{Solution}

\textbf{(1)} For isothermal expansion: $\Delta T = 0$, $\Delta U = 0$, and
\begin{equation}
\Delta S = R\frac{\Delta V}{V} > 0, \quad \Delta p = -\frac{p\Delta V}{V} < 0
\end{equation}

\textbf{(2)} For adiabatic expansion: $\Delta Q = 0$. Because the process proceeds very slowly it can be taken as a reversible process, then $\Delta S = 0$. The adiabatic process satisfies $pV^\gamma = \text{const}$. While $V$ increases, $p$ decreases ($\Delta p < 0$); and the internal energy decreases because work is done externally, thus $\Delta U < 0$, or $\Delta T < 0$.

\textbf{(3)} The process is equivalent to adiabatic free expansion of an ideal gas, thus $\Delta S > 0$, $\Delta U = 0$, $\Delta T = 0$, $\Delta p < 0$.

\textbf{(4)} The result is the same as that of isothermal free expansion, thus $\Delta T = 0$, $\Delta U = 0$, $\Delta S > 0$, $\Delta p < 0$.

\textbf{(5)} The result is the same as that of isothermal free compression, thus $\Delta T = 0$, $\Delta U = 0$, $\Delta p > 0$, $\Delta S < 0$.

%% PROBLEM 1053 %%
\section{Problem 1053}
\textbf{(MIT)}

A thermally insulated box is separated into two compartments (volumes $V_1$ and $V_2$) by a membrane. One of the compartments contains an ideal gas at temperature $T$; the other is empty (vacuum). The membrane is suddenly removed, and the gas fills up the two compartments and reaches equilibrium.

(a) What is the final temperature of the gas?

(b) Show that the gas expansion process is irreversible.

\subsection*{Solution}

\textbf{(a)} Freely expanding gas does no external work and does not absorb heat. So the internal energy does not change, i.e., $dW = 0$. The internal energy of an ideal gas is only a function of temperature; as the temperature does not change in the process:
\begin{equation}
\boxed{T_f = T}
\end{equation}

\textbf{(b)} Assuming a quasi-static process of isothermal expansion, we can calculate the change in entropy resulting from the free expansion. In the process, we have $dS = pdV/T$, $pV = NkT$. Hence:
\begin{equation}
S_f - S = \int dS = \int \frac{p}{T}dV = Nk\ln\frac{V_1 + V_2}{V_1} > 0
\end{equation}

Thus the freely expanding process of the gas is irreversible.

%% PROBLEM 1054 %%
\section{Problem 1054}
\textbf{(SUNY Buffalo)}

A thermally conducting, uniform and homogeneous bar of length $L$, cross section $A$, density $\rho$ and specific heat at constant pressure $c_p$ is brought to a nonuniform temperature distribution by contact at one end with a hot reservoir at temperature $T_H$ and at the other end with a cold reservoir at temperature $T_C$. The bar is removed from the reservoirs, thermally insulated and kept at constant pressure. Show that the change in entropy of the bar is:
\begin{equation}
\Delta S = C_p\left[\ln\frac{T_f^2}{T_H T_C} + \frac{T_H + T_C}{T_f} - 2\right]
\end{equation}
where $C_p = c_p \rho AL$, and $T_f = (T_H + T_C)/2$.

\subsection*{Solution}

As the temperature gradient in the bar is $(T_H - T_C)/L$, the temperature at the cross section at a distance $x$ from the end at $T_C$ can be expressed by:
\begin{equation}
T_x = T_C + (T_H - T_C)\frac{x}{L}
\end{equation}

As the bar is adiabatically removed, we have:
\begin{equation}
\int_0^L c_p \rho A \, T_x \, dx = c_p \rho A L \, T_f
\end{equation}

from which we obtain $T_f = (T_H + T_C)/2$.

The entropy change is:
\begin{equation}
\Delta S = \int_0^L c_p \rho A \ln\frac{T_f}{T_x} dx
\end{equation}

After integration:
\begin{equation}
\boxed{\Delta S = C_p\left[\ln\frac{T_f^2}{T_H T_C} + \frac{T_H + T_C}{T_f} - 2\right]}
\end{equation}

%% PROBLEM 1055 %%
\section{Problem 1055}
\textbf{(MIT)}

Show that, for an ideal gas, the entropy can be expressed as:
\begin{equation}
S = C_V \ln T + nR\ln V + \text{const}
\end{equation}

\subsection*{Solution}

For an ideal gas, from the first law:
\begin{equation}
dU = TdS - pdV
\end{equation}

Since $dU = nC_V dT$ and $p = nRT/V$:
\begin{equation}
dS = \frac{nC_V dT}{T} + \frac{nR\,dV}{V}
\end{equation}

Integrating:
\begin{equation}
\boxed{S = nC_V \ln T + nR\ln V + \text{const}}
\end{equation}

%% PROBLEM 1056 %%
\section{Problem 1056}
\textbf{(Wisconsin)}

A vessel is divided into two parts of equal volumes by a partition. One part is filled with an ideal gas A, the other with an ideal gas B. Both gases are at the same temperature $T$ and pressure $p$. Assuming that A and B are chemically inert and that the vessel is thermally insulated, calculate the change in entropy that occurs after the partition is removed and equilibrium is reestablished.

\subsection*{Solution}

When the partition is removed, each gas expands freely to fill the entire vessel. Since the gases are ideal and chemically inert, they behave independently.

For gas A, the entropy change is:
\begin{equation}
\Delta S_A = nR\ln\frac{2V}{V} = nR\ln 2
\end{equation}

Similarly for gas B:
\begin{equation}
\Delta S_B = nR\ln 2
\end{equation}

The total entropy change is:
\begin{equation}
\boxed{\Delta S = \Delta S_A + \Delta S_B = 2nR\ln 2}
\end{equation}

This is known as the \textbf{entropy of mixing}.

%% PROBLEM 1057 %%
\section{Problem 1057}
\textbf{(Columbia)}

An ideal monatomic gas at initial temperature $T_0$ is allowed to expand from $V_0$ to $2V_0$ in three different ways: (a) isothermally, (b) adiabatically, and (c) at constant pressure. For each case, calculate the work done by the gas, the heat added, and the changes in internal energy and entropy.

\subsection*{Solution}

For 1 mole of ideal monatomic gas: $C_V = \frac{3}{2}R$, $C_p = \frac{5}{2}R$, $\gamma = \frac{5}{3}$.

\textbf{(a) Isothermal expansion:}
\begin{align}
W &= RT_0\ln 2 \\
\Delta U &= 0 \quad \text{(isothermal)} \\
Q &= W = RT_0\ln 2 \\
\Delta S &= R\ln 2
\end{align}

\textbf{(b) Adiabatic expansion:}
\begin{equation}
T_f = T_0 \left(\frac{V_0}{2V_0}\right)^{\gamma-1} = T_0 \cdot 2^{-2/3}
\end{equation}
\begin{align}
Q &= 0 \\
\Delta U &= C_V(T_f - T_0) = \frac{3}{2}RT_0(2^{-2/3} - 1) \\
W &= -\Delta U = \frac{3}{2}RT_0(1 - 2^{-2/3}) \\
\Delta S &= 0 \quad \text{(reversible adiabatic)}
\end{align}

\textbf{(c) Constant pressure expansion:}
\begin{equation}
T_f = 2T_0 \quad \text{(since } V \propto T \text{ at constant } p\text{)}
\end{equation}
\begin{align}
W &= p\Delta V = RT_0 \\
\Delta U &= C_V \Delta T = \frac{3}{2}RT_0 \\
Q &= \Delta U + W = \frac{5}{2}RT_0 \\
\Delta S &= C_p\ln\frac{T_f}{T_0} = \frac{5}{2}R\ln 2
\end{align}

%% PROBLEM 1058 %%
\section{Problem 1058}
\textbf{(Berkeley)}

One mole of an ideal gas initially at $T_1 = 0^\circ$C is heated at constant volume to $T_2 = 100^\circ$C. What is the change in entropy?

\subsection*{Solution}

At constant volume:
\begin{equation}
\Delta S = \int_{T_1}^{T_2} \frac{C_V dT}{T} = C_V \ln\frac{T_2}{T_1}
\end{equation}

For one mole of ideal gas:
\begin{equation}
\boxed{\Delta S = C_V \ln\frac{373}{273}}
\end{equation}

For a monatomic gas ($C_V = \frac{3}{2}R$): $\Delta S = 3.86$ J/K

For a diatomic gas ($C_V = \frac{5}{2}R$): $\Delta S = 6.44$ J/K

%% PROBLEM 1059 %%
\section{Problem 1059}
\textbf{(Wisconsin)}

A resistor of 1000 $\Omega$ is connected to a 100 V battery for 10 s. The resistor is kept at a constant temperature of $27^\circ$C in a large bath.

(a) What is the change of entropy of the resistor?

(b) What is the change of entropy of the bath?

(c) What is the change of entropy of the system (resistor + bath)?

\subsection*{Solution}

\textbf{(a)} The resistor is at constant temperature, so:
\begin{equation}
\boxed{\Delta S_{\text{resistor}} = 0}
\end{equation}

\textbf{(b)} The heat generated in the resistor and absorbed by the bath is:
\begin{equation}
Q = \frac{V^2}{R}t = \frac{(100)^2}{1000} \times 10 = 100 \text{ J}
\end{equation}

The entropy change of the bath is:
\begin{equation}
\boxed{\Delta S_{\text{bath}} = \frac{Q}{T} = \frac{100}{300} = 0.333 \text{ J/K}}
\end{equation}

\textbf{(c)} The total entropy change of the system is:
\begin{equation}
\boxed{\Delta S_{\text{total}} = 0 + 0.333 = 0.333 \text{ J/K}}
\end{equation}

This is positive, confirming that the process is irreversible.

%% PROBLEM 1060 %%
\section{Problem 1060}
\textbf{(Columbia)}

Two equal quantities of the same ideal gas are initially at the same pressure $p$ but at different temperatures $T_1$ and $T_2$ ($T_1 > T_2$). They are then allowed to interact thermally at constant total volume. Find the change in total entropy.

\subsection*{Solution}

Let each quantity be $n$ moles. The final temperature is:
\begin{equation}
T_f = \frac{T_1 + T_2}{2}
\end{equation}

The entropy change for the first quantity (cooling from $T_1$ to $T_f$):
\begin{equation}
\Delta S_1 = nC_V \ln\frac{T_f}{T_1}
\end{equation}

The entropy change for the second quantity (heating from $T_2$ to $T_f$):
\begin{equation}
\Delta S_2 = nC_V \ln\frac{T_f}{T_2}
\end{equation}

Total entropy change:
\begin{equation}
\Delta S = \Delta S_1 + \Delta S_2 = nC_V \ln\frac{T_f^2}{T_1 T_2} = nC_V \ln\frac{(T_1 + T_2)^2}{4T_1 T_2}
\end{equation}

Since $(T_1 + T_2)^2 > 4T_1 T_2$ for $T_1 \neq T_2$:
\begin{equation}
\boxed{\Delta S = nC_V \ln\frac{(T_1 + T_2)^2}{4T_1 T_2} > 0}
\end{equation}
